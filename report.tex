\documentclass{article}

\title{RASPBERRY PI TRAINING IN PYTHON}
\author{OCHAKA SAMUEL}

\begin{document}
\maketitle

\section{Introduction}
This report is about the Raspberry Pi python training which was held in Makerere university at the college of computing and information sciences, GIS lab, level 3, block B. The training took five days starting from March 13th – 18th from 9am – 1pm every day. The training was organized by Python Software Foundation and was sponsored by Afrodjango initiative.

\section{Background}
Python software foundation in partnership with django software foundation, organizations comprising a diverse team of experienced lecturers, software developers, system analysts and integrators and open source software advocates who strongly believe in python, raspberry pi and Django framework has been organizing and training python programming and django framework to students from universities and post secondary institutions in the east African countries mainly in the first and second year. This time they chose Makerere University to train student form multidisciplinary fields in raspberry pi, python and django frame work.

\section{Body}
The following activities were carried out during the training:
\subsection{Introduction and motivation}
The first session was opened with the trainers introducing themselves and the trainees were also given time to introduce themselves and their courses. There shortly some words of inspirations were given by Mr. Osman who is from the Python software foundation.
\subsection{Raspberry Pi}
Basic information about the raspberry pi was given by Peter and the students were given the raspberry pi to look at and have a feel of it. The operation of the different components of the raspberry pi was also highlighted. Different modules were seen like raspberry pi 2 and raspberry pi 3 where the later had more features than the former like more gpio pins can be connected through Bluetooth and so on.
\subsection{Python and object oriented programming Basics}
This part was introduced on the second day by Mr. Osman. Python basics like the installations, python shell, syntax, operators, performing basic math (addition, subtraction, multiplication and division) was presented. The students were given simple exercises to complete and win prizes like T-shirts and airtime.
\subsection{Sensor /modules}
The students worked with different sensors and modules on the third day to know what each of them does and how to incorporate them into the raspberry pi. The sensors and modules examined include the following: LED module, Camera module, Buzzer, Sound module, Ultrasonic sensor, Rain sensor, Temperature sensor, RGB color sensor, Moisture sensor, etc.
\subsection{Bread board and raspberry pi programming}
On the fourth day, we students were introduced to the bread board components and how to program different sensors and modules integrated together in the same connection. This was based on event driven response where output in one sensor triggers (act as an input in another) sensor/module.
The students were then urged to try and different components of the own so as to learn and master how to work with the gpios and the raspberry pi.
\subsection{Django basic and web programming}
Django, a powerful python framework used for web programming was introduced to the students on Friday. The students were taken through the installation process, creating the first django website, running the server, deploying it to the browser and designing the web pages using CSS.
The students also formed groups and told to come up with creative and innovative ideas to work as their group project. The best group was awarded a prize of 100,000 Uganda shillings while the second group took 50,000 Uganda shillings.
The training was then concluded by the instructors announcing different that the team of students has got from clients and need to be worked upon. The projects included a smart house, a farm management system for remote monitoring of workers and all activities on a farm and finally a system for monitoring milk production in a dairy farm.


\section{Recommendation}
I recommend that python language should be included in the curriculum of computer science, software engineering, information technology, information systems as a core programming language right from year one.
The university should build more labs where these modules and components can be used by the students to build prototypes and real systems in order to solve community issues.
\section{Conclusion}
The training was really an “eye opener” to many of us while it was an exciting experience for many of the students to work with the real equipments that were being read in books to develop creative and innovative ideas.
\section{References}
http://www.afrodjangoinitiative.org/
\newline
https://www.python.org/psf/
\newline
https://www.sunfounder.com/sensor-v1.html

\end{document}
